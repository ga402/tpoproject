\documentclass[11pt]{article}
\usepackage[utf8]{inputenc}

\title{}
\author{}
\date{July 2018}

\begin{document}

\maketitle

\paragraph{Introduction:} Thrombopoietin is the major regulatory cytokine in the regulation of platelet homeostasis. The exact mechanisms behind its regulation are unknown. 
\paragraph{Methods:} 75 thrombocytopenic patients had blood samples taken between May to November 2014 in an outpatients setting. 67 patients had a diagnosis of ITP and 8 patients with a bone marrow-failure syndrome (4 aplastic anaemia, 3 post- BMT, 1 MDS). In total 130 samples were sent for TPO plasma levels with the majority of patients having duplicate samples. For the TPO analysis, blood samples were collected in sodium citrate tubes, double spun to remove platelet fractions and then stored at -80°C within 4hours of collection. Patients also have a full blood count measured at the time of collection. TPO levels measurements perfomed by quantitive sandwich enzyme immunoassay technique. For analysis, a bayesian regression model was used of (log-normalised) platelet count regressed against (log-normalised) TPO; $log(TPO) ~ alpha + beta*log(platelet)$. We used uninformative (flat) gaussian priors (mean 0, standard deviation 10) for the model. Results for this model were projected against different platelet counts (1, 10 & 150) to determine estimated TPO distributions.
\paragraph{Results:} Of the 130 samples collected, 30 failured to show any TPO. These samples were excluded as laboratory failures as repeated samples on the same patients found positive results. TPO concentrations ranged from 15 to 4572.5pg/mL and platelet counts ranged from 4 to 452 x10^9 /L. Our Bayesian regression model found there was a marked difference between patients with ITP and non-immune causes of thrombocytopenia (bone marrow failure). Furthermore, as the platelet count increased from 4 to 10 there was a significant change in TPO concentration, however over a platelet count of 10 there was little difference in TPO levels in either ITP or non-ITP cases. In the non-ITP cases the distribution predicted TPO concentrations in the non-ITP cases could be very wide even with higher platelet counts. We modelled the relationship between Log^10(TPO) concentrations and platelet count using a smoothing spline model. This also found in patients with ITP that after a platelet count of approximately 10 the TPO concentration in the blood stabilised. The regression smoothing spline model however indicated that for the the bone marrow failure syndromes however the TPO levels fell significantly. This is consistent with the bayesian model which showed a potentially very wide TPO distribution at higher platelet counts. This may also reflects the heterogeneity of this in terms of bone marrow pathology.
\paragraph{Conclusions:}


\paragraph{}
\textbf{Total characters: 3800}


\end{document}
